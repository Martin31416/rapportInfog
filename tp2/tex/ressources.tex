%!TEX encoding = IsoLatin

%
% Chapitre "Ressources"
%

\chapter{Ressources}
Comme notre application repr�sente un outil de dessin, on n'a pas utilis� beaucoup de ressources externes (des diff�rentes images, shaders, etc.). Les ressources utilis�es pour accomplir le travail sont les ''addons'' et la documentation officielle de ''Openframeworks'' (http://openframeworks.cc/documentation/). Pour construire l'interface graphique,on a utilis� le module ''ofxGui''. Cela nous a permis d'ajouter les diff�rents boutons, le menu et les composantes pour modifier les param�tres des primitives dessin�es. Pour lire et rendre les fichiers mod�les, on a utilis� le module ''ofxAssimpModelLoader'' (inclus en openframeworks aussi). La librairie ''OpenGl'' a �t� utilis�e surtout pour repr�senter l'ombre sur les mod�les (GL\_SMOOTH shading). Pour manipuler les entit�s 3D, on utilise ''ofxOpenCv''. De plus, pour ce qui est de la mise en place du Cube Map, le ''addon'' ofxCubeMap a �t� utilis�.
\\

On a utilis� �galement les librairies standard de C++ (<cmath>, <vector>) pour faire les op�rations math�matiques et pour utiliser des structures de donn�es qui contiennent des r�f�rences vers les objets instanci�s.