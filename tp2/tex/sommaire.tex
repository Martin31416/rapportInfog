%!TEX encoding = IsoLatin

%
% Chapitre "Introduction"
%

\chapter{Sommaire}

L'application Paint3D+ que nous d�veloppons est un programme d'�dition graphique interractif dans le style de PaintdotNet. Elle comporte 2 modes principaux: mode 2D et mode 3D.
%, en plus de comporter un mode pour les mod�les 3D 
\\

Concr�tement, l'application permet de construire des primitives vectorielles en 2D (ligne, triangle, rectangle et cercle), de contruire des primitives g�om�triques 3D et d'afficher un mod�le 3D. Tous ces �l�ments poss�dent des propri�t�s modifiables � partir de l'interface graphique.
Il est aussi possible d'importer des images et d'exporter le contenu de la sc�ne en image. De plus, il est possible de se d�placer dans la sc�ne � partir des cam�ras, d'appliquer diff�rents types de textures, telles que des textures d'�l�vation, de normale et de r�flexion et d'appliquer des courbes et surfaces param�triques. Finalement des effets post-process ainsi que des technique de lancer du rayon sont implant�es.
\\

Une interface intuitive est affich�e lors du d�marrage du programme et l'utilisateur peut interagir � l'aide des menus et des panaux graphiques. D'autres possibilit�s de manipulation sont aussi d�crites plus loins dans le rapport. 
\\

Toutes les entit�s g�om�triques sont organis�es dans une hi�rarchie de classes et elles peuvent �tre maniupul�es par diff�rentes m�thodes, telles que l'application de translations, rotations, changement d'�chelle, etc. Les primitives peuvent �tre aussi �tre compos�es dans une seule entit�.