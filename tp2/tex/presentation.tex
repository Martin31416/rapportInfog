%!TEX encoding = IsoLatin

%
% Chapitre "Pr�sentation"
%

\chapter{Pr�sentation}

L'�quipe de d�veloppement est compos� de 4 membres qui ont travaill� dans une symbiose fructueuse. Cela a permis d'effectuer un travail conforme aux exigences du cours. Plusieurs moyens de communication ont �t� utilis�s pour l'organisation de travail et la r�partition des t�ches. Pour la planification de travail, nous avons prioris� les rencontres en personnes tandis que la phase de construction de l'application a �t� r�alis�e surtout � l'aide des rencontres virtuelles (skype). �galement, la plateforme ''Github'' a �t� un outil indispensable pour le partae du code et la fusion des parties. � part le c�t� technique, la collaboration de tos les membres de l'�quipe a �t� essentielle. On a essay� d'int�grer le plus souvent les fonctionnalit�s et GitHub a �t� indispenpensable pour accomplir cela. De cette fa�on, on a r�duit le risque de conflits et on a d�tect� le plus vite possible les erreurs de d�veloppement (qui surviennent dans le cycle de vie du processus). L'int�gration continue implique de faire de petits changements au logiciel en lui appliquant des tests de qualit� (si les erreurs sont d�tect�es plus tard dans le processus de d�veloppement, leurs corrections sont plus difficiles).
\\

\noindent Les membres de l'�quipe et leur programme d'�tude:
\\

\noindent Gabriel Lefran�ois: g�nie informatique\\
Marcel Bernic: baccalaur�at en informatique\\
Martin Richard Cerda: g�nie informatique\\
William-Jos� Simard-Touzet: g�nie informatique

